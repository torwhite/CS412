% Options for packages loaded elsewhere
\PassOptionsToPackage{unicode}{hyperref}
\PassOptionsToPackage{hyphens}{url}
\PassOptionsToPackage{dvipsnames,svgnames,x11names}{xcolor}
%
\documentclass[
  letterpaper,
  DIV=11,
  numbers=noendperiod]{scrartcl}

\usepackage{amsmath,amssymb}
\usepackage{lmodern}
\usepackage{iftex}
\ifPDFTeX
  \usepackage[T1]{fontenc}
  \usepackage[utf8]{inputenc}
  \usepackage{textcomp} % provide euro and other symbols
\else % if luatex or xetex
  \usepackage{unicode-math}
  \defaultfontfeatures{Scale=MatchLowercase}
  \defaultfontfeatures[\rmfamily]{Ligatures=TeX,Scale=1}
\fi
% Use upquote if available, for straight quotes in verbatim environments
\IfFileExists{upquote.sty}{\usepackage{upquote}}{}
\IfFileExists{microtype.sty}{% use microtype if available
  \usepackage[]{microtype}
  \UseMicrotypeSet[protrusion]{basicmath} % disable protrusion for tt fonts
}{}
\makeatletter
\@ifundefined{KOMAClassName}{% if non-KOMA class
  \IfFileExists{parskip.sty}{%
    \usepackage{parskip}
  }{% else
    \setlength{\parindent}{0pt}
    \setlength{\parskip}{6pt plus 2pt minus 1pt}}
}{% if KOMA class
  \KOMAoptions{parskip=half}}
\makeatother
\usepackage{xcolor}
\setlength{\emergencystretch}{3em} % prevent overfull lines
\setcounter{secnumdepth}{-\maxdimen} % remove section numbering
% Make \paragraph and \subparagraph free-standing
\ifx\paragraph\undefined\else
  \let\oldparagraph\paragraph
  \renewcommand{\paragraph}[1]{\oldparagraph{#1}\mbox{}}
\fi
\ifx\subparagraph\undefined\else
  \let\oldsubparagraph\subparagraph
  \renewcommand{\subparagraph}[1]{\oldsubparagraph{#1}\mbox{}}
\fi


\providecommand{\tightlist}{%
  \setlength{\itemsep}{0pt}\setlength{\parskip}{0pt}}\usepackage{longtable,booktabs,array}
\usepackage{calc} % for calculating minipage widths
% Correct order of tables after \paragraph or \subparagraph
\usepackage{etoolbox}
\makeatletter
\patchcmd\longtable{\par}{\if@noskipsec\mbox{}\fi\par}{}{}
\makeatother
% Allow footnotes in longtable head/foot
\IfFileExists{footnotehyper.sty}{\usepackage{footnotehyper}}{\usepackage{footnote}}
\makesavenoteenv{longtable}
\usepackage{graphicx}
\makeatletter
\def\maxwidth{\ifdim\Gin@nat@width>\linewidth\linewidth\else\Gin@nat@width\fi}
\def\maxheight{\ifdim\Gin@nat@height>\textheight\textheight\else\Gin@nat@height\fi}
\makeatother
% Scale images if necessary, so that they will not overflow the page
% margins by default, and it is still possible to overwrite the defaults
% using explicit options in \includegraphics[width, height, ...]{}
\setkeys{Gin}{width=\maxwidth,height=\maxheight,keepaspectratio}
% Set default figure placement to htbp
\makeatletter
\def\fps@figure{htbp}
\makeatother

\KOMAoption{captions}{tableheading}
\makeatletter
\makeatother
\makeatletter
\makeatother
\makeatletter
\@ifpackageloaded{caption}{}{\usepackage{caption}}
\AtBeginDocument{%
\ifdefined\contentsname
  \renewcommand*\contentsname{Table of contents}
\else
  \newcommand\contentsname{Table of contents}
\fi
\ifdefined\listfigurename
  \renewcommand*\listfigurename{List of Figures}
\else
  \newcommand\listfigurename{List of Figures}
\fi
\ifdefined\listtablename
  \renewcommand*\listtablename{List of Tables}
\else
  \newcommand\listtablename{List of Tables}
\fi
\ifdefined\figurename
  \renewcommand*\figurename{Figure}
\else
  \newcommand\figurename{Figure}
\fi
\ifdefined\tablename
  \renewcommand*\tablename{Table}
\else
  \newcommand\tablename{Table}
\fi
}
\@ifpackageloaded{float}{}{\usepackage{float}}
\floatstyle{ruled}
\@ifundefined{c@chapter}{\newfloat{codelisting}{h}{lop}}{\newfloat{codelisting}{h}{lop}[chapter]}
\floatname{codelisting}{Listing}
\newcommand*\listoflistings{\listof{codelisting}{List of Listings}}
\makeatother
\makeatletter
\@ifpackageloaded{caption}{}{\usepackage{caption}}
\@ifpackageloaded{subcaption}{}{\usepackage{subcaption}}
\makeatother
\makeatletter
\@ifpackageloaded{tcolorbox}{}{\usepackage[many]{tcolorbox}}
\makeatother
\makeatletter
\@ifundefined{shadecolor}{\definecolor{shadecolor}{rgb}{.97, .97, .97}}
\makeatother
\makeatletter
\makeatother
\ifLuaTeX
  \usepackage{selnolig}  % disable illegal ligatures
\fi
\IfFileExists{bookmark.sty}{\usepackage{bookmark}}{\usepackage{hyperref}}
\IfFileExists{xurl.sty}{\usepackage{xurl}}{} % add URL line breaks if available
\urlstyle{same} % disable monospaced font for URLs
\hypersetup{
  pdftitle={Assignment 1},
  colorlinks=true,
  linkcolor={blue},
  filecolor={Maroon},
  citecolor={Blue},
  urlcolor={Blue},
  pdfcreator={LaTeX via pandoc}}

\title{Assignment 1}
\author{}
\date{}

\begin{document}
\maketitle
\ifdefined\Shaded\renewenvironment{Shaded}{\begin{tcolorbox}[breakable, frame hidden, sharp corners, borderline west={3pt}{0pt}{shadecolor}, interior hidden, boxrule=0pt, enhanced]}{\end{tcolorbox}}\fi

\usepackage{amsmath}

\begin{enumerate}
\def\labelenumi{\arabic{enumi}.}
\tightlist
\item
  For the following distribution, is A ⊥ B (i.e., A and B are
  independent)? (33 points)
\end{enumerate}

\begin{longtable}[]{@{}lll@{}}
\toprule()
a & b & P(A=a,B=b) \\
\midrule()
\endhead
0 & 0 & 0.5 \\
0 & 1 & 0.0 \\
1 & 0 & 0.0 \\
1 & 1 & 0.5 \\
\bottomrule()
\end{longtable}

If \(A\) \& \(B\) are independent then any of the following equations
hold:

\[
\displaylines{P(A|B) = P(A) \equiv \\ P(B|A) = P(B) \equiv \\P(A,B) = P(A)P(B)}
\]

First, find \(P(A)\) and \(P(B)\) \[
\displaylines{P(A) = P(A=1, B=0) + P(A=1, B=1) \\
     P(A) = 0.0 + 0.5 \\
     P(A) = 0.5 }
\]

\[
\displaylines{P(B) = P(A=0, B=1)+ P(A=1, B=1) \\
     P(B)= 0.0 + 0.5 \\
     P(B) = 0.5}
\]

So, to test independence we can take

\[
\displaylines{P(A,B) = P(A)P(B) \\
   0.5 = 0.5 * 0.5 \\
   0.5 \neq .25}
\]

So \(A\) \& \(B\) are \textbf{not} independent.

\newpage{}

\begin{enumerate}
\def\labelenumi{\arabic{enumi}.}
\setcounter{enumi}{1}
\tightlist
\item
  For the following distribution, is A ⊥ B\textbar C (i.e., A and B are
  conditionally independent given C)? (33 points)
\end{enumerate}

\begin{longtable}[]{@{}llll@{}}
\toprule()
a & b & c & P(A=a,B=b,C=c) \\
\midrule()
\endhead
0 & 0 & 0 & 0.056 \\
0 & 0 & 1 & 0.120 \\
0 & 1 & 0 & 0.224 \\
0 & 1 & 1 & 0.120 \\
1 & 0 & 0 & 0.024 \\
1 & 0 & 1 & 0.180 \\
1 & 1 & 0 & 0.180 \\
1 & 1 & 1 & 0.096 \\
\bottomrule()
\end{longtable}

\(A\) and \(B\) are conditionally independent given \(C\) if any holds:
\[
\displaylines{P(A|B,C) = P(A|C) \equiv \\ P(B|A,C) = P(B|C) \equiv \\P(A,B|C) = P(A|C)P(B|C)}
\]

First, \[
\displaylines{P(A) = P(A=1,B=0,C=0) + P(A=1,B=0,C=1) + P(A=1,B=1,C=0) + P(A=1,B=1,C=1) \\
P(A) = .024 + .180 +.180 +.096 = .516}
\]

Look at :
https://stats.libretexts.org/Bookshelves/Probability\_Theory/Applied\_Probability\_(Pfeiffer)/05\%3A\_Conditional\_Independence/5.01\%3A\_Conditional\_Independence

\newpage{}

\begin{enumerate}
\def\labelenumi{\arabic{enumi}.}
\setcounter{enumi}{2}
\tightlist
\item
  Consider two binary random variables A and B. If A ⊥ B (i.e., A and B
  are independent), and P(A = 0, B = 0) = 0.18 and P(A = 1, B = 0) =
  0.28, what is the probability of P(A = 0, B = 1)? (34 points)
\end{enumerate}

\begin{longtable}[]{@{}lll@{}}
\toprule()
a & b & P(A=a,B=b) \\
\midrule()
\endhead
0 & 0 & 0.18 \\
0 & 1 & n.a. (\(x\)) \\
1 & 0 & 0.28 \\
1 & 1 & n.a. (\(y\)) \\
\bottomrule()
\end{longtable}

From the given probabilities we know:

\[
\displaylines{P(B=0) = P(A=0,B=0) + P(A=1,B=0) \\
P(B=0) = .18 + .28 \\
P(B=0) = .46}
\]

Hence, \[
\displaylines{P(B=1) = 1- P(B=0) \\
P(B=1) = 1 - .46 \\
P(B=1) = .54 \\}
\]

**look back at lecture notes for bernoulli random variables



\end{document}
